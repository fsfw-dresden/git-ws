\documentclass{beamer}
% \includeonlyframes{uj}
\synctex=1
\usepackage[utf8]{inputenc}
\usepackage[ngerman]{babel}
\usepackage[T1]{fontenc}
\usepackage{graphicx}
\usepackage[absolute,overlay]{textpos}
\usepackage{csquotes}
\usepackage{eurosym}
\usepackage{wasysym}
\MakeAutoQuote{„}{”}
\usepackage[absolute,overlay]{textpos} % for \textblock

\usepackage{lmodern}


% \usepackage{lmodern}
% \usepackage{anyfontsize}

\hypersetup{
  colorlinks=true,
  linkcolor=[rgb]{1, 1, 1}, %
  urlcolor=[rgb]{.2, .2, .5} %
  % allcolors=[rgb]{0, 0, 0} % schwarz
}

\usetheme{Dresden}
\setbeamertemplate{navigation symbols}{}
\setbeamertemplate{footline}[frame number]

\title{Versionsverwaltung mit git: Warum und wie.}
\subtitle{Bunter Nachmittag des iFSR, 02.10.2017}

\addtobeamertemplate{frametitle}{}{%
  \begin{textblock*}{130mm}(.963\textwidth,8.4mm)
    \includegraphics[width=2.25cm]{img-src/fsfw-logo.pdf}
  \end{textblock*}
}

\begin{document}

\begin{frame}
  \begin{center}%
    \includegraphics[width=3cm]{img-src/fsfw-logo-with-text.pdf}\\

    \vspace*{-0.5\baselineskip}

    \parbox{.95\columnwidth}{\centering\Large\inserttitle}

    \vspace*{\baselineskip}

    \structure{\large \insertsubtitle}
  \end{center}
\end{frame}

%%%%%%%%%%%%%%%%%%%%%%%%%%%%%%%%%%%%%%%%%%%%%%%%%%%%%%%%%%%%%%%%%%%%%%%%%%%%%%%%
\begin{frame}[label=wb]

\begin{center}
 \vspace{10mm}
\structure{\Huge  <Eigenwerbung>}
\end{center}
 
\end{frame}


%%%%%%%%%%%%%%%%%%%%%%%%%%%%%%%%%%%%%%%%%%%%%%%%%%%%%%%%%%%%%%%%%%%%%%%%%%%%%%%%

\begin{frame}[label=ct1]
  \frametitle{Wer sind wir?}

  \onslide<+->

  \begin{itemize}
  \item Hochschulgruppe an der TU (gegründet 2014, ca. 10 P.)
  \item Studierende (TU, HTW) und andere Leute
  \item Hochschulen als Zielgruppe (Multiplikationswirkung)\\
    und Arbeitsfeld (Räume, Strukturen)

    \bigskip\onslide<+->

  \item Bisherige Projekte
    \begin{itemize}
    \item Linux-Install-Party, Linux-Presentation-Day
    \item Verschlüsselungsgewinnspiel
    \item Monatliche Sprechstunde zu \LaTeX{} u.a.
    \item Formulierung eines Programmpapiers
    \item „Uni-Stick”:~80 $\times$ 8\,GB mit freier Software
    \end{itemize}
  \end{itemize}
\end{frame}

%%%%%%%%%%%%%%%%%%%%%%%%%%%%%%%%%%%%%%%%%%%%%%%%%%%%%%%%%%%%%%%%%%%%%%%%%%%%%%%%

\begin{frame}[label=ct2]
  \frametitle{Warum machen wir das? Aus Überzeugung!}

  \onslide<+->

  \begin{itemize}
  \item \emph{Überzeugung 1}: freie und quelloffene Software ist (oft) besser\\
    (technische + nicht technische Argumente)\\
    \bigskip\onslide<+->
  \item \emph{Überzeugung 2}: öffentlich finanzierte wissenschaftliche Inhalte
    (AutorInnen, GutachterInnen) sollten nicht von öffentlich finanzierten
    Bibliotheken für horrende Summen von Zeitschriften-Verlagen gekauft werden
    müssen
  \end{itemize}
\end{frame}

%%%%%%%%%%%%%%%%%%%%%%%%%%%%%%%%%%%%%%%%%%%%%%%%%%%%%%%%%%%%%%%%%%%%%%%%%%%%%%%%

\begin{frame}[label=ct1b,t]
  \frametitle{Projekt Uni-Stick}

  \begin{itemize}
  \item \textbf{4000} Flyer in Ersti-Tüten: \textbf{Gutscheine} für 8\,GB Stick mit freier Software
    fürs Studium, 550 \geneuro{} vom TU-StuRa für 80 Stk.
  \item Live-Linux / freie Windows-Programme
    \pause
  \item Hat viel Arbeit gemacht
    \pause
  \item Ist gut angekommen (ca. 250 TN)
  \end{itemize}

  \begin{textblock*}{40mm}[0.,0.](80mm,37mm)
    \visible<2->{
      \includegraphics[width=40mm]{img-src/usb-hub}
    }
  \end{textblock*}

  \begin{textblock*}{55mm}[0.,0.](15mm,50mm)
    \visible<3->{
      \includegraphics[width=55mm]{img-src/uni-stick-ausgabe-vortrag}
    }
  \end{textblock*}

  \begin{textblock*}{55mm}[0.,0.](71mm,63mm)
    \visible<4->{
      \begin{itemize}
       \item Accessibility:
       \hspace{-5mm}
       \begin{itemize}
        \item brltty
        \item gnome-orca (Screenreader)
        \item ...
        \item WIP!
       \end{itemize}

       
       
      \end{itemize}

    }
  \end{textblock*}
  
\end{frame}

%%%%%%%%%%%%%%%%%%%%%%%%%%%%%%%%%%%%%%%%%%%%%%%%%%%%%%%%%%%%%%%%%%%%%%%%%%%%%%%%

\begin{frame}[label=ct3]
  \frametitle{Zukunftsideen}

  \begin{itemize}
  \item Fortführung "`Uni-Stick"'
  \item Studierende zum Nutzen/Verbessern freier Software animieren
    \begin{itemize}
    \item Mehr Blog-Beiträge
    \item Kurse (\LaTeX / Python /\textbf{ Git} / Inkscape / \dots)
    \item Ansible-Infrastruktur-Stipendium
    \item OpenSource-Wettbewerb/Preis
    \item \dots
    \end{itemize}

    \bigskip

  \item Aufmerksamkeit erzeugen / Lobby-Arbeit

    \bigskip

  \item Vernetzung mit anderen Städten

  \end{itemize}

\end{frame}

%%%%%%%%%%%%%%%%%%%%%%%%%%%%%%%%%%%%%%%%%%%%%%%%%%%%%%%%%%%%%%%%%%%%%%%%%%%%%%%%

\begin{frame}[label=ct4]
  \frametitle{Weitere Informationen}

  \onslide<+->

  \begin{center}
    \url{https://fsfw-dresden.de/}
    % HACK THE PLANET!
    $\;\;\left\{\;\;\text{
        \parbox{2.3cm}{
          \texttt{uni-stick}\\[1mm]
          \texttt{blog}\\[1mm]
          \texttt{newsletter}
          \texttt{mitmachen}\\[1mm]
          \texttt{fork}\\[-3mm]
        }}
    \right.$

    \vspace*{2\bigskipamount}

    \includegraphics[width=50mm]{img-src/fsfw-netzwerke}
  \end{center}

\end{frame}

%%%%%%%%%%%%%%%%%%%%%%%%%%%%%%%%%%%%%%%%%%%%%%%%%%%%%%%%%%%%%%%%%%%%%%%%%%%%%%%%
\begin{frame}[label=wb2]

\begin{center}
 \vspace{10mm}
\structure{\Huge  </Eigenwerbung>}
\end{center}
 
\end{frame}

%%%%%%%%%%%%%%%%%%%%%%%%%%%%%%%%%%%%%%%%%%%%%%%%%%%%%%%%%%%%%%%%%%%%%%%%%%%%%%%%

\begin{frame}[label=ol1]
  \frametitle{Gliederung}
  \begin{itemize}
   \item Warum Versionsverwaltung
   \item Warum git
   \item Git Einführung (mit Praxis)
  \end{itemize}

\end{frame}

%%%%%%%%%%%%%%%%%%%%%%%%%%%%%%%%%%%%%%%%%%%%%%%%%%%%%%%%%%%%%%%%%%%%%%%%%%%%%%%%
\begin{frame}[label=why10]
\frametitle{Warum Versionsverwaltung?}

\begin{itemize}
  \item Projekte bestehen aus schrittweisen Änderungen
  \item Bedürfnis, zu vorherigem Zustand zurückkehren zu können
  
  \begin{itemize}
\item  ("`Savegame"')
    \end{itemize}

  \item Naiver Ansatz: \hspace{10mm}
  \raisebox{-0.5\height}{\includegraphics[width=40mm]{img-src/my-software-dirs}}
  
  \item Probleme:
  \begin{itemize}
   \item Speicherplatz
   \item Fehlende Übersicht
   \item Skaliert nicht (Teamwork)
  \end{itemize}

  \end{itemize}
\end{frame}

%%%%%%%%%%%%%%%%%%%%%%%%%%%%%%%%%%%%%%%%%%%%%%%%%%%%%%%%%%%%%%%%%%%%%%%%%%%%%%%%
% TODO: image source http://techidiocy.com/understand-git-clone-command-svn-checkout-vs-git-clone/
% https://softwareengineering.stackexchange.com/questions/136079/are-there-any-statistics-that-show-the-popularity-of-git-versus-svn
% http://www.wastedpotential.com/version-control-throwdown-git-vs-svn/

\begin{frame}[label=why20]
\frametitle{Warum Git? (1)}

\begin{itemize}
  \item Lösung 1: zentrale Versionsverwaltung
  \begin{itemize}
   \item CVS (1986), SVN (2000)
   \item Idee:\\[-2mm]
   \hspace{2cm}\includegraphics[width=30mm]{img-src/structure-svn1}
   
  \end{itemize}
  \item Probleme:
  \begin{itemize}
   \item Abhängig von Server-Erreichbarkeit
   \item Performanz
  \end{itemize}

  \end{itemize}
\end{frame}

%%%%%%%%%%%%%%%%%%%%%%%%%%%%%%%%%%%%%%%%%%%%%%%%%%%%%%%%%%%%%%%%%%%%%%%%%%%%%%%%
\begin{frame}[label=why30]
\frametitle{Warum Git? (2)}

\begin{itemize}
  \item Lösung 2: \textbf{de}zentrale Versionsverwaltung
  \begin{itemize}
   \item mercurial (2005), bazaar (2005) \textbf{git} (2005)
   \item Idee:\\[-2mm]
   \hspace{2cm}\includegraphics[width=30mm]{img-src/structure-git1}
  \end{itemize}
  \item Vorteile:
  \begin{itemize}
   \item Alle Operationen lokal $\rightarrow$ schnell, unabhängig
   \item Einfaches "`branching"' und "`merging"'
   \item ...
  \end{itemize}
  \end{itemize}
\end{frame}

%%%%%%%%%%%%%%%%%%%%%%%%%%%%%%%%%%%%%%%%%%%%%%%%%%%%%%%%%%%%%%%%%%%%%%%%%%%%%%%%
\begin{frame}[label=why40]
\frametitle{Warum Git? (3)}

\begin{textblock*}{\textwidth}[0.5,0.](60mm,20mm)
\begin{center}%
\only<1>{
\includegraphics[width=80mm]{img-src/svn-vs-git-boxing}
}
\only<2->{
\includegraphics[width=80mm]{img-src/svn-vs-git-stats}
}
~\\[3mm]
\visible<3->
{
2017: Git = defacto Standard
}
\end{center}
\end{textblock*}


\end{frame}

%%%%%%%%%%%%%%%%%%%%%%%%%%%%%%%%%%%%%%%%%%%%%%%%%%%%%%%%%%%%%%%%%%%%%%%%%%%%%%%%

\begin{frame}[label=gitintro10]
\frametitle{Einführung in git}
\begin{itemize}
 \item Wir empfehlen: git Bedienung via Kommandozeile
 \item Syntax: \texttt{git <command> [<args>]}
 \item Beispiele:
 
 \begin{itemize}
  \item \texttt{git init}
  \item \texttt{git add myscrip.py}
  \item \texttt{git commit -m "{}add basic functionality"}
  \item \texttt{git push}\\[2mm]
  \pause
  \item \texttt{git status}
  \item \texttt{git log}
  \item \texttt{git branch develop}
  \item \texttt{git checkout master}
  \item \texttt{git merge develop}
  \item \texttt{git blame myscrip.py}
  \item \texttt{git diff}
  \item \texttt{git difftool}
  
 \end{itemize}
\end{itemize}

\begin{textblock*}{\textwidth}[0.5,0.](120mm, 60mm)
 
\visible<3->{
\begin{itemize}
 \item[] ~
\begin{itemize}

  \item \texttt{git clone}
  \item \texttt{git help}
  \item \texttt{git rebase} \\[2mm]
  \item \texttt{git config}
  
  \item \texttt{gitk}

\end{itemize}
\end{itemize}
}
\end{textblock*}

\end{frame}

%%%%%%%%%%%%%%%%%%%%%%%%%%%%%%%%%%%%%%%%%%%%%%%%%%%%%%%%%%%%%%%%%%%%%%%%%%%%%%%%

\begin{frame}[label=gitintro20]
\frametitle{Praxis 1: Erste Schritte}
\begin{itemize}
 \item Konfiguration anpassen
 \begin{itemize}
  \item \texttt{git config -{}-global user.email "foo@bar.de"}
  \item \texttt{git config -{}-global user.name "Your Name"}
  ...
 \end{itemize}

 \item Eigenes Repo erstellen
 \begin{itemize}
  \item \texttt{git init}
 \end{itemize}
 \item Alternativ: Bestehenedes Repo klonen
 \begin{itemize}
  \item \texttt{git clone <url>}
 \end{itemize}
 \item[]
 \pause
 \item Hintergrund: Wo speichert git die relevanten Informationen?
 \item[$\rightarrow$] Verstecktes Verzeichnis:
 \raisebox{-0.5\height}{
 \includegraphics[width=20mm]{img-src/git-dir1}}
 
 
\end{itemize}

\end{frame}
%%%%%%%%%%%%%%%%%%%%%%%%%%%%%%%%%%%%%%%%%%%%%%%%%%%%%%%%%%%%%%%%%%%%%%%%%%%%%%%%

\begin{frame}[label=gitintro30]
\frametitle{Theorie: typischer Ablauf / "`staging area"'}
\vspace{-11mm}
\begin{center}
\includegraphics[width=115mm]{img-src/git-add-commit-push-workflow}\\[-2mm]

\pause
\includegraphics[width=80mm]{img-src/git-staging-area}
\end{center}
Wozu zweiphasiger Commit-Prozess?
\begin{itemize}
 \item Ermöglicht präzise, hoch aufgelöste Commits\\(z.B. nur bestimmte Änderungen einer Datei)
 \item[$\rightarrow$] nachvollziehbare aussagekräftige Commit-History
\end{itemize}


\end{frame}

%%%%%%%%%%%%%%%%%%%%%%%%%%%%%%%%%%%%%%%%%%%%%%%%%%%%%%%%%%%%%%%%%%%%%%%%%%%%%%%%
\begin{frame}[label=gitintro40]
\frametitle{Praxis: Änderungen committen}
\begin{itemize}
 \item[] \texttt{echo "{}Hallo Welt"{} > README.md}
 \item[] \texttt{git status}
 \item[] \texttt{git add README.md}
 \item[] \texttt{git status}
 \item[] \texttt{git commit -m "{}New content ofREADME"{}}
 \item[] \texttt{git status}
 \bigskip
 
 \pause 
 \item Sich Überblick verschaffen
 \item[] git log
 \item Commit-Nachricht korrigieren (mit Editor)
 \item[]  \texttt{git config -{}-global core.editor mcedit}
 \item[] \texttt{git commit --amend}
\end{itemize}
 
\end{frame}


\begin{frame}[label=link10]{Tipps und Links}
Weitere Ressourcen:
\tiny
\begin{itemize}
 \item 
\url{https://learn.adafruit.com/an-introduction-to-collaborating-with-version-control/initializing-a-repository-and-making-commits}
\item ...
\end{itemize}

 
\end{frame}





%%%%%%%%%%%%%%%%%%%%%%%%%%%%%%%%%%%%%%%%%%%%%%%%%%%%%%%%%%%%%%%%%%%%%%%%%%%%%%%%
\begin{frame}[label=uj]{Ausblick}
\begin{itemize}
 \item Fragen?\\[10mm]
 \item 
 \begin{itemize}
  \item Enigmail (Plugin für Thunderbird)
  \item Mailvelope (Browserplugin für Webmail)
  \item Zeit?\\[5mm]
  \pause
  \end{itemize}
  \item Unterstützung später {\tiny (im Rahmen unserer Möglichkeiten)}:
  \begin{itemize}
   \item \url{https://fsfw-dresden.de/sprechstunde}
   \item \url{kontakt@fsfw-dresden.de}
 \end{itemize}
 
\end{itemize}



\end{frame}

%%%%%%%%%%%%%%%%%%%%%%%%%%%%%%%%%%%%%%%%%%%%%%%%%%%%%%%%%%%%%%%%%%%%%%%%%%%%%%%%
\end{document}
