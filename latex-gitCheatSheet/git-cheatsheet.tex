%
% Copyright (c) 2012-2013 Max Oberberger (max@oberbergers.de)
%
% This program is free software: you can redistribute it and/or modify
% it under the terms of the GNU General Public License as published by
% the Free Software Foundation, either version 3 of the License, or
% (at your option) any later version.
% 
% This program is distributed in the hope that it will be useful,
% but WITHOUT ANY WARRANTY; without even the implied warranty of
% MERCHANTABILITY or FITNESS FOR A PARTICULAR PURPOSE. See the 
% GNU General Public License for more details.
% 
% You should have received a copy of the GNU General Public License 
% along with this program.  If not, see <http://www.gnu.org/licenses/>.
%

\documentclass[a4paper,fontsize=7.5pt]{scrreprt}
%\documentclass[a4paper,fontsize=7.5pt, ngerman]{scrreprt}

% Packages
\usepackage[utf8]{inputenc}
%\usepackage[ngerman]{babel}
%\usepackage[]{babel}
\usepackage[T1]{fontenc}
\usepackage{textcomp}
\usepackage{tabularx}
\usepackage{enumerate}
\usepackage{multicol}
\usepackage{vmargin}
\usepackage{hyperref}
\usepackage{graphicx}

\usepackage[table]{xcolor}

\setmarginsrb{1.1cm}{1.0cm}% left, top
		      {1.0cm}{1.0cm}% right, down
                	      {7mm}{0.5cm}% head: height, distance
	               {7mm}{0.5cm}% foot: height, distance

% Set column space
\setlength{\columnsep}{0.25em}

% Define colours – todo: use fsfw-colour scheme
\definecolorset{hsb}{}{}{red,0,.4,0.95;orange,.1,.4,0.95;green,.25,.4,0.95;yellow,.15,.4,0.95}
\definecolorset{hsb}{}{}{blue,.55,.4,0.95;purple,.7,.4,0.95;pink,.8,.4,0.95;blue2,.58,.4,0.95}
\definecolorset{hsb}{}{}{magenta,.9,.4,0.95;green2,.29,.4,0.95}

% Redefine sections
\makeatletter
\renewcommand{\section}{\@startsection{section}{1}{0mm}
	{-1.7ex}{0.7ex}{\normalfont\large\bfseries}}
\renewcommand{\subsection}{\@startsection{subsection}{2}{0mm}
	{-0.0ex}{0.5ex}{\normalfont\normalsize\bfseries}}
\makeatother

% No section numbers
\setcounter{secnumdepth}{0}

% set Version
\newcommand{\Version}[0]{1.0}

% No indentation
\setlength{\parindent}{0em}

% No header and footer
\pagestyle{empty}

% set own shortcuts
\newcommand{\sourcetext}[1] {
	\begin{tabularx}{\hsize}{X}
			{\texttt{} #1}
	\end{tabularx}
}

\newcommand{\cmdline}[1] {\sourcetext{\$ #1}}

\newcommand{\colouredbox}[2] {
	\colorbox{#1!40}{
		\begin{minipage}{0.95\linewidth}
			{
			\rowcolors[]{1}{#1!20}{#1!10}
			#2
			}
		\end{minipage}
	}
        \vspace{1mm}
}

\newcommand{\annotation}[1] {
  \parbox{\textwidth}{\raggedleft #1}\par
}

%%%%%%%%%%%%%%%%%%%%%%%%%%%%%%%%%%%
%%%%%%%%%% DOCUMENT
%%%%%%%%%%%%%%%%%%%%%%%%%%%%%%%%%%%
\begin{document}
\vspace*{-1cm}
\hfill\includegraphics[width=0.1\textwidth]{../img-src/fsfw-logo-with-text.pdf}
\vspace*{-1cm}\begin{center}{\large\bfseries Git-Spickzettel Nr. 1}\end{center}
\begin{multicols}{2}
%% configure your settings for git
\colouredbox{yellow}{
	\begin{center}\section{Git konfigurieren}\end{center}
	\cmdline{git config [-{}-global] [option]}
	\annotation{mit \texttt{-{}-global}: wird es gespeichert in \texttt{$\tilde{\ }$/.gitconfig}}
	\subsection{Information über den Anwender}
		\sourcetext{user.name NAME}
		\sourcetext{user.email EMAIL}
	\subsection{Farbige Ausgabe einschalten}
		\sourcetext{color.ui auto}
        \subsection{Verbesserung der interaktiven Nutzererfahrung}
                \sourcetext{interactive.singlekey true}
}

%% Create git Repositories
\colouredbox{magenta}{
	\begin{center}\section{Erstellen und klonen von Repositorien}\end{center}
	\subsection{Von existierenden Daten}
		\cmdline{cd \ my\_project\_dir}\\
		\cmdline{git init}\\
		\cmdline{git add .}
                \cmdline{git commit -m 'initial commit'}
	\subsection{Von existierendem Repo}
		\cmdline{git clone path/to/existing/repo path/to/new/repo}\\
		\cmdline{git clone you@host.de:dir/project.git}\\
		\cmdline{git clone http://[USER@]host.de/project.git}
}

%% Browse through files/repository
\colouredbox{purple}{
	\begin{center}\section{Informationen erhalten}\end{center}
	\subsection{Veränderte Dateien im Arbeitsverzeichnis}
		\cmdline{git status}
	\subsection{Änderungen an überwachten Dateien}
		\cmdline{git diff}
	\subsection{Änderungen zwischen ID1 und ID2}
		\cmdline{git diff $<$ID1$>$ $<$ID2$>$}
	\subsection{Historie der Änderungen}
		\cmdline{git log}
                \cmdline{gitk}
}

%% Commit your changes
\colouredbox{red}{
	\begin{center}\section{Arbeiten mit dem Index und committen (dt.: einreichen) von Änderungen}\end{center}
        \subsection{Füge alle Änderungen in einer Datei oder Verzeichnis zum Index hinzu}
                \cmdline{git add path/to/add}
        \subsection{Wähle interaktiv alle Änderungen zum Hinzufügen/ Committen aus}
                 \cmdline{git add -p [path/to/preselect/changes]}
                 \cmdline{git commit -p}
        \subsection{Committe alle hinzugefügten Änderungen zum Index}
                \cmdline{git commit}
	\subsection{Füge alle lokalen Änderungen hinzu und committe sie}
		\cmdline{git commit -a}
	\subsection{Committe Änderungen mit direkter Angabe einer Änderungsnachricht}
		\cmdline{git commit -m "<message>"}
}

%% How to handle branches
\colouredbox{orange}{
	\begin{center}\section{Arbeiten mit Branches (dt.: Äste)}\end{center}
	\subsection{Alle Branches auflisten}
		\cmdline{git branch}
                \cmdline{git branch -a}
                \annotation{Listet auch entfernte* Branches auf}
	\subsection{Zu einem Branch wechseln}
		\cmdline{git checkout <branch>}
	\subsection{Branch B1 in B2 mergen (dt. mischen)}
		\cmdline{git checkout <B2>}
		\cmdline{git merge <B1>}
                \annotation{oder}
		\cmdline{git merge -{}-no-ff <B1>}
		\annotation{Erstelle einen Commit auch wenn fast-forwarding möglich ist}
	\subsection{Erstellen eines Branches basierend auf HEAD und checkout}
		\cmdline{git checkout -b <branch>}
	\subsection{Löschen eines Branches}
		\cmdline{git branch -d <branch>}
		\cmdline{git branch -D <branch>}
                \annotation{Lösche einen, noch nicht mit dem Standard-Branch gemergeten, Branch}
}


%% how to update repository
\colouredbox{pink}{
	\begin{center}\section{Änderungen von entferntem Repos holen}\end{center}
	\subsection{Lade Änderungen von remote *(einem entfernten Repo) herunter}
		\cmdline{git fetch [remote]}
                \annotation{\texttt{remote} ist voreingestellt auf: \texttt{origin}}
        \subsection{Änderungen bekommen}
                \cmdline{git pull [remote] [refspec]}
                \annotation{\texttt{remote} defaults to \texttt{origin}}
}


%% Resolve merge conflicts
% \colouredbox{gray}{
% 	\begin{center}\section{Lösen von Merge-Konflikten}\end{center}
% 	\subsection{Merge-Konflikte anzeigen}
% 		\cmdline{git diff}
% 	\subsection{Zeige Merge-Konflikte gegen Basisdateien}
% 		\cmdline{git diff -{}-base <FILE>}
% 	\subsection{Zeige Merge-Konflikte gegen Änderungen Anderer}
% 		\cmdline{git diff -{}-theirs <FILE>}
% 	\subsection{Zeige Merge-Konflikte gegen deine Änderungen}
% 		\cmdline{git diff -{}-ours <FILE>}
% 	\subsection{Nach dem Auflösen von Konflikten committen nicht vergessen}
% 		\cmdline{git add <CONFLICTING\_FILE>}\\
%                 \cmdline{git commit}
% }

%% Publish your changes
\colouredbox{blue}{
	\begin{center}\section{Änderungen veröffentlichen}\end{center}
	\subsection{Änderungen nach remote pushen (also in ein entferntes Repo schieben)}
		\cmdline{git push [origin] [branch]}
	\subsection{Einen tag (dt.: ???) erstellen}
		\cmdline{git tag [-s] <tag name>}
                \annotation{mit \texttt{-s} den tag mit GPG signieren}
	\subsection{Einen Patch vorbereiten}
		\cmdline{git format-patch origin}
}

%% Revert changes
\colouredbox{green}{
	\begin{center}\section{Änderungen rückgängig machen}\end{center}
	\subsection{Zum zuletzt committeten Zustand zurückkehren}
		\cmdline{git reset -{}-hard}
	\subsection{Einen bestimmten Commit rückgängig machen}
		\cmdline{git revert <ID>}
                \annotation{Dies ist sicher für bereits veröffentlichte Commits}
	\subsection{Korrigieren/ Ändern des letzten Commits}
		\cmdline{git commit -{}-amend}
                \annotation{Tue dies \emph{niemals} mit bereits veröffentlichten Commits\\
                  außer du weißt genau, was du tust}
}

%% Git tips to handle things
\colouredbox{gray}{
	\begin{center}\section{Verschiedenes}\end{center}
	\subsection{Dokumentation/ Hilfe}
		\cmdline{git help [command]}
		\cmdline{man git-[command]}
	\subsection{Einen Branch löschen (lokal und remote)}
		\cmdline{git branch -d <branch>}
		\cmdline{git push <origin> :<branch>}
	\subsection{Die unveröffentlichte Historie interaktiv aufhübschen}
		\cmdline{git rebase -i <ID>}
}
\end{multicols}
\begin{center}
\footnotesize
\rule{0.9\linewidth}{0.25pt}
\par FSFW git-Spickzettel \url{https://github.com/fsfw-dresden/git-ws} – Datum: \today \ – Version \Version\ \\
basierend auf \url{https://github.com/chiemseesurfer/latex-gitCheatSheet} Version 1.5
von Max Oberberger (\texttt{github@oberbergers.de})
\end{center}

%% @TODO: Eine Illustration zur Staging-Area & Co. sowie auch zu Best-Practice wie z.B.: Git-Flow/ Git-Dit fände ich gut für die Rückseite.

\end{document}
