%
% Copyright (c) 2012-2013 Max Oberberger (max@oberbergers.de)
% Copyright (c) 2017 fsfw Dresden (kontakt@fsfw-dresden.de)
%
% This program is free software: you can redistribute it and/or modify
% it under the terms of the GNU General Public License as published by
% the Free Software Foundation, either version 3 of the License, or
% (at your option) any later version.
%
% This program is distributed in the hope that it will be useful,
% but WITHOUT ANY WARRANTY; without even the implied warranty of
% MERCHANTABILITY or FITNESS FOR A PARTICULAR PURPOSE. See the
% GNU General Public License for more details.
%
% You should have received a copy of the GNU General Public License
% along with this program.  If not, see <http://www.gnu.org/licenses/>.
%
\documentclass[a4paper,fontsize=7.5pt]{scrreprt}
\synctex=1

% Packages
\usepackage[utf8]{inputenc}
\usepackage[T1]{fontenc}
\usepackage{textcomp}
\usepackage[english]{babel}
\usepackage{tabularx}
\usepackage{enumerate}
\usepackage{multicol}
\usepackage{vmargin}
\usepackage{graphicx}
\usepackage{lmodern}
\usepackage[table]{xcolor}

\usepackage[pdftex]{hyperref}

\definecolor{myblue}{HTML}{4040dd}
\hypersetup{
    pdftitle = {FSFW git CheatSheet},
    colorlinks = true,
    linkcolor={myblue},
    citecolor={myblue},
    urlcolor={myblue},
}



\setmarginsrb{1.1cm}{8mm}% left, top
		      {1.0cm}{1.0cm}% right, down
                	      {7mm}{0.5cm}% head: height, distance
	               {7mm}{0.5cm}% foot: height, distance

% Set column space
\setlength{\columnsep}{0.25em}

% original colors (quit high saturation)
% Define colours – todo: use fsfw-colour scheme
% \definecolorset{hsb}{}{}{red,0,.4,0.95;orange,.1,.4,0.95;green,.25,.4,0.95;yellow,.15,.4,0.95}
% \definecolorset{hsb}{}{}{blue,.55,.4,0.95;purple,.7,.4,0.95;pink,.8,.4,0.95;blue2,.58,.4,0.95}
% \definecolorset{hsb}{}{}{magenta,.9,.4,0.95;green2,.29,.4,0.95;gray2,.6,.04,0.85}

% new colors (quit low saturation, save printer-toner, (looks much better on paper than on screen))
\definecolorset{hsb}{}{}{red,0,.025,0.95;orange,.1,.025,0.95;green,.25,.025,0.95;yellow,.15,.025,0.95}
\definecolorset{hsb}{}{}{blue,.55,.025,0.95;purple,.7,.025,0.95;pink,.8,.025,0.95;blue2,.58,.025,0.95}
\definecolorset{hsb}{}{}{magenta,.9,.025,0.95;green2,.29,.025,0.95;gray2,.6,.01,0.85}

% \definecolor{gray2}{gray}{0.7}

% Redefine sections
\makeatletter
\renewcommand{\section}{\@startsection{section}{1}{0mm}
	{-1.7ex}{0.7ex}{\normalfont\large\bfseries}}
\renewcommand{\subsection}{\@startsection{subsection}{2}{0mm}
	{-0.0ex}{0.5ex}{\normalfont\normalsize\bfseries}}
\makeatother

% No section numbers
\setcounter{secnumdepth}{0}

% set Version
\newcommand{\Version}[0]{1.1}

% No indentation
\setlength{\parindent}{0em}

% No header and footer
\pagestyle{empty}

% set own shortcuts
\newcommand{\sourcetext}[1] {
	\begin{tabularx}{\hsize}{X}
			{\tt #1}
	\end{tabularx}
}

\newcommand{\cmdline}[1] {\sourcetext{\$ #1}}

\newcommand{\colouredbox}[2] {
	\colorbox{#1!25}{
		\begin{minipage}{0.95\linewidth}
			{
			\rowcolors[]{1}{#1!10}{#1!10}
			#2
			}
		\end{minipage}
	}
        \vspace{1mm}
}

\newcommand{\annotation}[1] {
  \parbox{\textwidth}{\raggedleft #1}\par
}

%%%%%%%%%%%%%%%%%%%%%%%%%%%%%%%%%%%
%%%%%%%%%% DOCUMENT
%%%%%%%%%%%%%%%%%%%%%%%%%%%%%%%%%%%
\begin{document}
\parbox{7cm}{\vspace*{-3cm}Hochschulgruppe für Freie Software und Freies Wissen \\
\textcolor{myblue}{\url{https://fsfw-dresden.de}} \hspace{0.85em} kontakt@fsfw-dresden.de} \hfill\includegraphics[width=0.1\textwidth]{../img-src/fsfw-logo-with-text.pdf}
\vspace*{-1cm}\begin{center}{\large\bfseries FSFW Git Workshop Cheat Sheet}\end{center}
\begin{multicols}{2}
%% configure your settings for git
\colouredbox{gray2}{
\textbf{Note}: Parts of commands in [square brackets] are optional.
}

\colouredbox{yellow}{
	\begin{center}\section{Configuring git}\end{center}
	\cmdline{git config [-{}-global] [option]}
	\annotation{with \texttt{-{}-global} it is stored in \texttt{$\tilde{\ }$/.gitconfig}}
	\subsection{Information about the user}
		\cmdline{git config user.name NAME}
		\cmdline{git config user.email EMAIL@HOST.TLD}
	\subsection{Enabling coloured output}
		\cmdline{git config color.ui auto}
        \subsection{Improve interactive user-experience}
                \cmdline{git config interactive.singlekey true}
}

%% Create git Repositories
\colouredbox{magenta}{
	\begin{center}\section{Creating or cloning repositories}\end{center}
	\subsection{From existing data}
		\cmdline{cd \ my\_project\_dir}\\
		\cmdline{git init}\\
		\cmdline{git add .}
                \cmdline{git commit -m 'initial commit'}
	\subsection{From existing repo}
		\cmdline{git clone path/to/existing/repo path/to/new/repo}\\
		\cmdline{git clone you@host.de:dir/project.git}\\
		\cmdline{git clone http://[USER@]host.de/project.git}
}

%% Browse through files/repository
\colouredbox{purple}{
	\begin{center}\section{Getting information}\end{center}
	\subsection{Files changed in working directory}
		\cmdline{git status}
	\subsection{Changes of tracked files}
		\cmdline{git diff}
	\subsection{Changes between ID1 and ID2}
		\cmdline{git diff $<$ID1$>$ $<$ID2$>$}
	\subsection{History of changes}
		\cmdline{git log}
         \cmdline{gitk}
         \cmdline{gitk -{}-all}
}

%% Commit your changes
\colouredbox{red}{
	\begin{center}\section{Working with the Index and committing changes}\end{center}
        \subsection{Add all changes in a file or directory to the index}
                \cmdline{git add path/to/file\_or\_dir}
                \cmdline{git add .}
        \subsection{Interactively select changes for addition}
                 \cmdline{git add -p [path/to/preselect\_some\_files]}

                 % removed because not important enough
                 % \cmdline{git commit -p}

        \subsection{Commit changes added to the index}
                \cmdline{git commit}
	\subsection{Add and commit all local changes}
		\cmdline{git commit -a}
	\subsection{Commit changes without editing the message}
		\cmdline{git commit -m "<message>"}
}

%% How to handle branches
\colouredbox{orange}{
	\begin{center}\section{Working with Branches}\end{center}
	\subsection{List all branches}
		\cmdline{git branch}
                \cmdline{git branch -a}
                \annotation{list remote branches as well}
	\subsection{Switch to a branch}
		\cmdline{git checkout <branch>}
	\subsection{Merge Branch B1 into B2}
		\cmdline{git checkout <B2>}
		\cmdline{git merge <B1>}
                \annotation{or}
		\cmdline{git merge -{}-no-ff <B1>}
		\annotation{generates commit even if fast-forwarding is possible}
	\subsection{Create Branch based on HEAD and checkout}
		\cmdline{git checkout -b <branch>}
	\subsection{Delete a branch}
		\cmdline{git branch -d <branch>}
		\cmdline{git branch -D <branch>}
                \annotation{Delete a branch that is not merged to the default branch}
}


%% how to update repository
\colouredbox{pink}{
	\begin{center}\section{Get changes from upstream}\end{center}
	\subsection{Fetch changes from a remote}
		\cmdline{git fetch [remote]}
                \annotation{\texttt{remote} defaults to \texttt{origin}}
        \subsection{Get changes}
                \cmdline{git pull [remote] [refspec]}
                \annotation{\texttt{remote} defaults to \texttt{origin}}
}


%% Resolve merge conflicts
% \colouredbox{gray}{
% 	\begin{center}\section{Resolve merge conflicts}\end{center}
% 	\subsection{View merge conflicts}
% 		\cmdline{git diff}
% 	\subsection{View merge conflicts against base file}
% 		\cmdline{git diff -{}-base <FILE>}
% 	\subsection{View merge conflicts against other changes}
% 		\cmdline{git diff -{}-theirs <FILE>}
% 	\subsection{View merge conflicts against your changes}
% 		\cmdline{git diff -{}-ours <FILE>}
% 	\subsection{After resolving conflicts, merge with}
% 		\cmdline{git add <CONFLICTING\_FILE>}\\
%                 \cmdline{git commit}
% }

%% Publish your changes
\colouredbox{blue}{
	\begin{center}\section{Publishing changes}\end{center}
	\subsection{Push changes to a remote}
        \cmdline{git push [origin] [branch]}
		\cmdline{git push}
	\subsection{Create tags}
		\cmdline{git tag [-s] <tag name>}
                \annotation{with \texttt{-s} the tag is signed with GPG}
	\subsection{Prepare a patch}
		\cmdline{git format-patch origin}
}

%% Revert changes
\colouredbox{green}{
	\begin{center}\section{Reverting changes}\end{center}
	\subsection{Return to last committed state}
		\cmdline{git reset -{}-hard}
	\subsection{Revert specific commit}
		\cmdline{git revert <ID>}
                \annotation{this is safe for published commits}
	\subsection{fix/change last commit}
		\cmdline{git commit -{}-amend}
                \annotation{\emph{never} do this on a published commit\\
                  unless you know what you are doing}
}

%% Git tips to handle things
\colouredbox{gray2}{
	\begin{center}\section{Misc}\end{center}
	\subsection{Documentation/help}
		\cmdline{git help [command]}
		\cmdline{man git-<command>}
	\subsection{Delete branch (locally and remote)}
		\cmdline{git branch -d <branch>}
		\cmdline{git push <origin> :<branch>}
	\subsection{Beautify non-pushed history interactively}
		\cmdline{git rebase -i <ID>}
}
\end{multicols}
\begin{center}
\footnotesize
\rule{0.9\linewidth}{0.25pt}
\par FSFW git CheatSheet – Version \Version ~~\textcolor{myblue}{\url{https://fsfw-dresden.de/git-ws}}\\
\noindent adapted from \url{https://github.com/chiemseesurfer/latex-gitCheatSheet} Version 1.5 by Max Oberberger (\texttt{github@oberbergers.de})\par
\noindent licensed under GPLv3+
\end{center}
\end{document}
